\section*{Introduction}
\addcontentsline{toc}{section}{Introduction}

Dans un contexte sociologique d'individualisme croissant lié à une considération en baisse des origines généalogiques directes et du passé de l'humanité au profit de l'avenir, la transmission oratoire perpétrée par les contes est en déclin.

Parallèlement, la pratique du jeu de rôle, jeu de société comportant des similitudes certaines avec le genre du conte, a su se développer efficacement au sein de cette société.\\


À des fins d'analyse dudit phénomène, cet écrit portera tout d'abord son étude sur la nature profonde des contes oraux\footnote{Dans un souci de complétude de l'étude menée, le terme \textit{conte} ultérieurement utilisé ne fera référence qu'aux contes oraux} dans le but d'appréhender les causes de l'affaiblissement de leur pratique.

Dans un second temps, le principe du jeu de rôle sera détaillé suivi du dépassement de son caractère ludique dans le but d'expliciter les facteurs clés de son succès.

Les vergences de ces pratiques seront enfin étudiées dans l'optique de déterminer leurs potentielles relations, et plus spécifiquement si le jeu de rôle peut-être considéré comme un \textit{conte interactif}.

\clearpage
