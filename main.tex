\documentclass[a4paper,12pt,final,oneside]{article}

\usepackage[frenchb]{babel}
\usepackage[utf8]{inputenc}

\setlength{\parskip}{1ex} %Definit l'espace entre les paragraphes
\pagestyle{empty}

\title{Contes et jeux de rôles}
\author{Rémi Domingues}
\date{\today}

\pagestyle{plain}
\bibliographystyle{plain}

\begin{document}

%Title page
\maketitle
\clearpage


%Sommaire
\thispagestyle{empty}
\tableofcontents
\clearpage


\section{Introduction}
\setcounter{page}{1}
\clearpage


\section{Le conte, vecteur de transmission d'humanisme en déclin}
\subsection{Porteur d'expérience}
\cite{nouss2003conteur}
« Le premier véritable récit est et demeure le conte », Walter Benjamin, Der Erzähler\cite{benjamin1991gesammelte}
Conteur = narrateur\\
Conte opposé au mythe\\
Littérature orale\\
Anciennement et encore aujourd'hui, le conte était un vecteur de perpétuation de l'expérience et de renouvellement de la tradition
La transmission orale via le conte est en perte de vitesse (Walter Benjamin) car elle est essentiellement relative à la transmission d'une expérience, dont l'intérêt porté est en net déclin.\\
Conteurs : sédentaire (perpétuant un lointain temporel constitué de riches expériences passées) / voyageur (rapportant un lointain spatial d'expériences vécues ou entendues en d'autres lieux)\\

L'expérience ici mentionnée peut faire référence à un récit véridique ou bien façonné, porteur d'une morale visant à enrichir l'auditoire du passé, dans le but de guider celui-ci dans les choix futurs qui se poseront à lui.\\
Par ailleurs, le conteur est dans un rapport de proximité avec son auditoire, ce qui permet l'exposé de la distance.\\
Or, cette recherche d'expérience est en perte de vitesse, notamment de par l'information journalistique informant tout un chacun des derniers évènements survenus, prémachés, analysés et expliqués. Au contraire, le conte laisse l'auditoire tirer ses propres conclusions. Le conte délivre en effet un fait nu, exempte de toute explication. La valeur du conte ne réside plus alors dans son intrigue, mais dans la mise en valeur de l'acte de vivre une expérience à proprement parler. C'est donc une conception de l'être humain qui est ainsi véhiculée, celle de vivre des expériences, de traverser des épreuves, et de les surmonter.\\

\subsection{L'écoute flottante (à détailler ?)}
Dans l'écoute du conte, le procédé freudien d'attention flottante consistant en l'absence d'attention dirigée ou focalisée, permettrait à l'auditeur de s'oublier lui-même, afin que «les mots qu'il entend[e] s'inscrivent profondément en lui» (ibid), facteur essentiel de l'assimilation et de la mémorisation des contes par l'auditoire via une écoute détendue.\\

\subsection{Le conte et la mort (reformuler)}
«Le parfait récit naît de l'accumulation de ses versions successives». Chaque conteur apose en effet sa marque au récit, transformant celui-ci de par son vécu.\\
L'existence d'un conte réside dans la mémoire de son auditoire, sa survie dans la transmission perpétuée par ses conteurs. Un conte meurt s'il n'est pas transmis.\\
Une cause cause annexe de déclin pourrait résider dans la conception de la mort par l'humain.\\
Selon W. B., le conteur tient de la mort son autorité. En effet, les contes étant faits de vécu, une vie d'expériences fondera donc une sagesse ou un savoir dont la transmission sera d'autant plus riche que la vie en fut emplie :  «La mort est la sanction de tout ce que relate le conteur». À l'heure de sa mort, toute personne devient donc digne d'être écoutée.\\

Le siècle dernier aurait néanmoins bouleversé la conception sociale de la mort dans nos moeurs. Les progrès technologiques et les atrocités commises ont eu pour effet l'acceptation de la conception d'une mort en masse (Hiroshima), pleine de souffrances (camps de déportation), banalisée et quotidienne (médias).\\
La mort étant transformée, l'autorité qu'on en retire en tant que conteur en est également impactée.\\
Les évènements passés ayant bouleversé le rapport mort/vie, ceux-ci ont également impacté la conception que tout un chacun se fait de la survie. La mort d'un être humain conte revêt alors une importance bien moindre, de même que celle d'un conte.
\clearpage


\section{Le jeu de rôle, un renouveau de création populaire}
\cite{caira2007jeux}
\subsection{Présentation JDR}
«Le principe des jeux de role se rapproche davantage du théâtre improvisé que d'un jeu de société traditionnel» \cite{cristofari2010lecteur}\\
Un Maître du jeu et entre trois et huit joueurs, chaque joueur interprétant un personnage imaginaire. Le maître du jeu, un narrateur, est en charge de choisir ou inventer un monde, une trame aventuresque brodée à l'oral, dans laquelle évolueront les joueurs. Il conte, décrit, fait survenir des évènements à sa convenance, devant également réagit aux actions des joueurs. Il s'agît pour les joueurs de résoudre des énigmes, combattre en équipe (mécanismes de dés afin d'évaluer la réussite d'une action) et vivre une expérience interactive reposant majoritairement sur l'imagination.\\

Les univers dont sont inspirés les mondes et scénarios ont souvent une origine littéraire (Call of Cthulhu, Lovecraft => ref, Le seigneur des anneaux, Tolkien) ou cinématographique (Star Wars).\\
La fidélité à l'oeuvre est alors libre au maître du jeu. On peut ainsi observer des fictions en accord irréprochable avec l'oeuvre dont elle sont issues, comme d'autres s'en inspirant vaguement, voire inventant un monde depuis ses fondements. Il en est de même pour le respect aux règles des jeux de rôles.\\
De nombreux livres de règles ont en effet été édités, chacun véhiculant avec lui la description étendue d'un monde, des races (humain, nain, elfe...) et classes (catégorie, telle magicien, prêtre, épéiste, chevalier...) de personnages, mais également des règles de jeu et un set de scénarios prédéfinis. Le respect de ces ouvrages est également laissé libre à chacun.\\

\subsection{L'attrait des joueurs}
L'attrait des joueurs pour les jeux de rôle proviet essentiellement de l'identification des joueurs au personnage que ceux-ci jouent, souvent plus importante que l'histoire vécue, pouvant entraîner une forte addiction, du même acabit que celle pouvant être constatée sur des jeux

Le but du jeu de rôle est aussi d'étendre l'intrigue d'une oeuvre de fiction, 

\subsection{Psychanalyse}
Le jeu de role est en outre utilisé dans le domaine de la psychanalyse, considéré comme une aire d'expérience où les réactions de l'individu vis-à-vis de son environnement peuvent être analysées. 


trans-372 P4 paragraphe 9 <==============================================================================




\section{Le jeu de rôle, un conte interactif}
«L'art de raconter des histoires est toujours l'art de reprendre celles que l'on a entendues»

Différence entre conte et jeu de role : \\
Personnages joueurs interagissant dans une histoire que le maître du jeu déroule pour eux, celui-ci s'adaptant à ses joueurs
Univers souvent imaginaire
Rarement porteur de morales, mais plutôt d'aventures

Similitudes :\\
Conteur et Maître du jeu
Le plus important, c'est l'ambiance


Je souhaitais donc en premier lieu rapprocher le conteur, voyageur, musicien à ses heures et perpétuant une tradition orale, au maître de jeu, à la vie souvent bien différente mais jouant également un rôle social et transportant ses auditeurs dans un monde imaginaire.
Les jeux de rôle pourraient donc alors être assimilés à des descendants des contes, rapprochant auditeurs et conteurs et ouvrant les esprits.\\

Mon analyse se porterait donc en premier lieu sur le métier du conteur, fort éloigné de celui de maître du jeu, puis se recentrerait sur les similitudes que l'on peut noter entre l'écoute d'un conte et la pratique du jeu de rôle, ce dernier étendant les possibilités du conte par son interaction de l'auditoire.

\clearpage


\section{Conclusion}
\clearpage

\section{Bibliographie}
\def\section*#1{} % Supprime le titre "References" et l'espace qui lui etait alloue
\bibliography{references}

% http://www.joueurs-de-chimeres.net/decouvrez-le-jeu-de-role,le-jeu-de-roles-heritier-moderne-des-contes,1,3
% http://trans.revues.org/372
% http://strenae.revues.org/94
% http://www.ffjdr.org/le-jeu-de-role/definitions-du-jeu-de-role

\end{document}