\section*{Conclusion}
\addcontentsline{toc}{section}{Conclusion}

TODO : lettre ae\\
Le jeu de rôle est donc une sorte de conte interactif
TODO : étayer en relisant le pph (son aspect novateur, ce qu'il apporte en plus du conte, pourquoi c'est chouette d'être roliste...)
Conte = vecteur d'expérience, dont le cours est en baisse
Jeux de rôle = vivre une expérience !
Le jeu de rôle serait donc un digne successeur des contes, réagissant face à la dévaluation de l'expérience en profitant de SOMETHING.

Perte de pouvoir avec la conception changeante de la mort et celle de l'expérience

C'est une évolution, un genre est né du conte, libéré d'anciens objectifs (morale, tradition, vécu) pour se concentrer sur le plaisir du joueur

On ne fait pas que raconter l'histoire (le conteur a tous les roles), on la vit (on les répartis)

Jdr > contes de par le personnage adapté au joueur et son rôle prépondérant dans le déroulement de l'histoire. Une interactivité croissante alliée à des exercices intellectuels pour changer de registre de langue et de profondeur du jeu favorisent l'immersion et l'attachement à cette pratique. Il revêt donc plus d'intérêt.

JDR = conte interactif


\clearpage
