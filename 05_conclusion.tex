\section*{Conclusion}
\addcontentsline{toc}{section}{Conclusion}

Porteur de savoir, d'expériences et de traditions, le conte est de nos jours en déclin. Malgré une communauté fidèle, la baisse du cours de l'expérience alliée à une conception changeante de la mort ont progressivement raison de son autorité littéraire.

Par le caractère central accordé à l'identité et aux désirs de son auditoire, le jeu de rôle a su se répandre dans notre société. Alliant interactions et variations de registre de langue et de profondeur du jeu, celui-ci favorise une immersion exaltante, un attachement profond aux protagonistes décrits et une passion certaine dans sa pratique.

Le jeu de rôle serait donc un digne successeur des contes, et bien plus aujourd'hui qu'un simple conte interactif. Accessible à tous, le jeu de rôle se voit certes libéré de visées morales et traditionnelles, mais endosse la mission non moins délicate de vecteur thérapeutique divertissant de construction sociale et psychologique.

\clearpage
